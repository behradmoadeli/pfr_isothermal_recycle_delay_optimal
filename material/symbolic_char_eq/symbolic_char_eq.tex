% (
%     + Dv^2.exp(lt+g)
%     + 2kv^2.exp(lt+g)
%     - 2lv^2.exp(lt+g)
%     - Dv^2.exp(lt+f)
%     - Dp.exp(lt+g)
%     - 2kv^2.exp(lt+f)
%     + 2lv^2.exp(lt+f)
%     + Dp.exp(lt+f)
%     +2v.sqrt(p)[
%         + k.exp(lt+g)
%         - l.exp(lt+g)
%         + k.exp(lt+f)
%         - l.exp(lt+f)
%         - 2Rk.exp(f).exp(g)
%         + 2Rl.exp(f).exp(g)
%     ]
% )
%     /
%     (4(k - l).sqrt(p))

% if p = 0:
%     (
%         4*D^2*exp(l*t)*exp((v - v*(1 - D^4)^(1/2))/(2*D))*(-(D^2 + 1)*(D - 1)*(D + 1))^(1/2) - 4*D^2*exp(l*t)*exp((v + v*(1 - D^4)^(1/2))/(2*D))*(-(D^2 + 1)*(D - 1)*(D + 1))^(1/2) + 4*D^6*exp(l*t)*exp((v + v*(1 - D^4)^(1/2))/(2*D))*(-(D^2 + 1)*(D - 1)*(D + 1))^(1/2) - 4*D^6*exp(l*t)*exp((v - v*(1 - D^4)^(1/2))/(2*D))*(-(D^2 + 1)*(D - 1)*(D + 1))^(1/2) - 2*v^2*exp((v + v*(-(D^2 + 1)*(D - 1)*(D + 1))^(1/2))/(2*D))*exp(l*t)*(1 - D^4)^(1/2) + 2*v^2*exp((v - v*(-(D^2 + 1)*(D - 1)*(D + 1))^(1/2))/(2*D))*exp(l*t)*(1 - D^4)^(1/2) + D^4*R*v^2*exp((v + v*(-(D^2 + 1)*(D - 1)*(D + 1))^(1/2))/(2*D))*exp((v + v*(1 - D^4)^(1/2))/(2*D)) - D^4*R*v^2*exp((v + v*(-(D^2 + 1)*(D - 1)*(D + 1))^(1/2))/(2*D))*exp((v - v*(1 - D^4)^(1/2))/(2*D)) - D^4*R*v^2*exp((v - v*(-(D^2 + 1)*(D - 1)*(D + 1))^(1/2))/(2*D))*exp((v + v*(1 - D^4)^(1/2))/(2*D)) + D^4*R*v^2*exp((v - v*(-(D^2 + 1)*(D - 1)*(D + 1))^(1/2))/(2*D))*exp((v - v*(1 - D^4)^(1/2))/(2*D)) - D^8*R*v^2*exp((v + v*(-(D^2 + 1)*(D - 1)*(D + 1))^(1/2))/(2*D))*exp((v + v*(1 - D^4)^(1/2))/(2*D)) + D^8*R*v^2*exp((v + v*(-(D^2 + 1)*(D - 1)*(D + 1))^(1/2))/(2*D))*exp((v - v*(1 - D^4)^(1/2))/(2*D)) + D^8*R*v^2*exp((v - v*(-(D^2 + 1)*(D - 1)*(D + 1))^(1/2))/(2*D))*exp((v + v*(1 - D^4)^(1/2))/(2*D)) - D^8*R*v^2*exp((v - v*(-(D^2 + 1)*(D - 1)*(D + 1))^(1/2))/(2*D))*exp((v - v*(1 - D^4)^(1/2))/(2*D)) + 2*D^4*v^2*exp((v + v*(-(D^2 + 1)*(D - 1)*(D + 1))^(1/2))/(2*D))*exp(l*t)*(1 - D^4)^(1/2) - 2*D^4*v^2*exp((v - v*(-(D^2 + 1)*(D - 1)*(D + 1))^(1/2))/(2*D))*exp(l*t)*(1 - D^4)^(1/2) - R*v^2*exp((v + v*(-(D^2 + 1)*(D - 1)*(D + 1))^(1/2))/D)*(1 - D^4)^(1/2)*(-(D^2 + 1)*(D - 1)*(D + 1))^(1/2) - R*v^2*exp((v - v*(-(D^2 + 1)*(D - 1)*(D + 1))^(1/2))/D)*(1 - D^4)^(1/2)*(-(D^2 + 1)*(D - 1)*(D + 1))^(1/2) + R*v^2*exp((v + v*(-(D^2 + 1)*(D - 1)*(D + 1))^(1/2))/D)*(1 - D^4)^(1/2)*(-(D^2 + 1)*(D - 1)*(D + 1))^(3/2) + R*v^2*exp((v - v*(-(D^2 + 1)*(D - 1)*(D + 1))^(1/2))/D)*(1 - D^4)^(1/2)*(-(D^2 + 1)*(D - 1)*(D + 1))^(3/2) - 2*v^2*exp((v + v*(-(D^2 + 1)*(D - 1)*(D + 1))^(1/2))/(2*D))*exp(l*t)*(1 - D^4)^(1/2)*(-(D^2 + 1)*(D - 1)*(D + 1))^(1/2) - 2*v^2*exp((v - v*(-(D^2 + 1)*(D - 1)*(D + 1))^(1/2))/(2*D))*exp(l*t)*(1 - D^4)^(1/2)*(-(D^2 + 1)*(D - 1)*(D + 1))^(1/2) + 2*D^4*v^2*exp((v + v*(-(D^2 + 1)*(D - 1)*(D + 1))^(1/2))/(2*D))*exp(l*t)*(1 - D^4)^(1/2)*(-(D^2 + 1)*(D - 1)*(D + 1))^(1/2) + 2*D^4*v^2*exp((v - v*(-(D^2 + 1)*(D - 1)*(D + 1))^(1/2))/(2*D))*exp(l*t)*(1 - D^4)^(1/2)*(-(D^2 + 1)*(D - 1)*(D + 1))^(1/2) + 2*R*v^2*exp((v + v*(-(D^2 + 1)*(D - 1)*(D + 1))^(1/2))/(2*D))*exp((v - v*(-(D^2 + 1)*(D - 1)*(D + 1))^(1/2))/(2*D))*(1 - D^4)^(1/2)*(-(D^2 + 1)*(D - 1)*(D + 1))^(1/2) + 2*R*v^2*exp((v + v*(-(D^2 + 1)*(D - 1)*(D + 1))^(1/2))/(2*D))*exp((v - v*(-(D^2 + 1)*(D - 1)*(D + 1))^(1/2))/(2*D))*(1 - D^4)^(1/2)*(-(D^2 + 1)*(D - 1)*(D + 1))^(3/2))
%         /(
%             4*v*(D^2 + 1)*(1 - D^4)^(1/2)*(D - 1)*(D + 1)*(-(D^2 + 1)*(D - 1)*(D + 1))^(1/2))

\documentclass{article}

\usepackage{amsmath}
% Preamble
\title{Characteristic Equation}
\author{Brad Moadeli}
\date{\today} % or specify a specific date

\begin{document}

\maketitle % Creates the title page based on the information in the preamble

The characteristic equation will look like the following:

\begin{multline}
    \frac{e^{(\lambda t+\frac{v}{2D})}\sinh{(\frac{\sqrt{v^2-4D\left(k-\lambda\right)}}{2D})}(v^2+2D^2)}{\sqrt{v^2-4D\left(k-\lambda\right)}}\\
    - \frac{v\sqrt{v^2-4D\left(k-\lambda\right)}\left(Re^{(\frac{v}{D})}-\cosh{(\frac{\sqrt{v^2-4D\left(k-\lambda\right)}}{2D})}\right)}{\sqrt{v^2-4D\left(k-\lambda\right)}} = 0
\end{multline}

The denominator is the same on both sides. Therefore, we solve for the numerators first to obtain a potential solutoin:

\begin{multline}
    e^{(\lambda t+\frac{v}{2D})}\sinh{(\frac{\sqrt{v^2-4D\left(k-\lambda\right)}}{2D})}(v^2+2D^2)\\
    - v\sqrt{v^2-4D\left(k-\lambda\right)}\left(Re^{(\frac{v}{D})}-\cosh{(\frac{\sqrt{v^2-4D\left(k-\lambda\right)}}{2D})}\right) = 0
\end{multline}

and then check the following:

\begin{enumerate}
    \item The denominator does not go to zero close to the obtained solution. In this case, the potential solution is a solution for the original equation.
    \item The denominator goes to zero close to the obtained solution. In this case, the following must be checked:
\end{enumerate}

\begin{equation}
    e^{(\lambda t+\frac{v}{2D})}\left(D+2\left(k-\lambda\right)\right) - v\left(Re^{(\frac{v}{D})}-1\right) = 0
\end{equation}

which is the $\lim_{\lambda \rightarrow \lambda_0}\frac{N(\lambda)}{D(\lambda)}$, where $\lambda_0$ is the potential solution. $\lambda_0$ is the solution to the original equation only if the above expression holds true.

\end{document}
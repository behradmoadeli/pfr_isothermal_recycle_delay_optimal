\subsection{Adjoint Operator}

The adjoint operator $\mathfrak{A}^*$ plays a critical role in analyzing the spectral properties of the system. Although the operator $\mathfrak{A}$ is not self-adjoint, the combined eigenmodes of $\mathfrak{A}$ and $\mathfrak{A}^*$ may still form a bi-orthonormal basis, typical of a Riesz-spectral operator \autocite{curtainbook}. The adjoint operator $\mathfrak{A}^*$ is given by:

\begin{equation} \label{eq:adjoint_A}
    \begin{aligned}
        {\mathfrak{A}}^{*} =&
        \begin{bmatrix}
            D \partial_{\zeta \zeta} + v \partial_\zeta +k_r & 0\\
            0 & -\frac{1}{\tau} \partial_\zeta
        \end{bmatrix}\\
        D(\mathfrak{A}^*) =& \Bigl\{ y = [y_1, y_2]^T \in Y:
        y(\zeta), \partial_\zeta y(\zeta), \partial_{\zeta \zeta} y(\zeta) \quad \mathrm{a.c.},\\
        &D \partial_\zeta y_1(1) + v y_1(1) = \frac{1}{\tau} y_2(1),\ R v y_1(0) = \frac{1}{\tau} y_2(0),\ \partial_\zeta y_1(0) = 0 \Bigr\}
    \end{aligned}
\end{equation}

Given that $\mathfrak{A}$ is not self-adjoint, we examine the spectral properties to determine if $\mathfrak{A}$ is Riesz-spectral by solving the characteristic equations of both $\mathfrak{A}$ and $\mathfrak{A}^*$.

\subsection{Eigenvalue Problem}

The eigenvalue problem for $\mathfrak{A}$ is formulated as:

\begin{equation} \label{eq:eig_prob}
        \mathfrak{A} \Phi_i(\zeta) = \lambda_i \Phi_i(\zeta)
\end{equation}

where $\lambda_i \in \mathbb{C}$ is the $i^{\text{th}}$ eigenvalue, and $\Phi_i(\zeta) = [\phi_{i,1}(\zeta), \phi_{i,2}(\zeta)]^T$ is the corresponding eigenfunction. The system of PDEs is reduced to the following ODE system:

\begin{equation} \label{eq:eigval_calc_2}
    \partial_\zeta \begin{bmatrix}
        \phi_1 \\ \partial_\zeta \phi_1 \\ \phi_2
    \end{bmatrix} = \begin{bmatrix}
        0 & 1 & 0 \\
        \frac{\lambda-k_r}{D} & \frac{v}{D} & 0 \\
        0 & 0 & \tau \lambda 
    \end{bmatrix} \begin{bmatrix}
        \phi_1 \\ \partial_\zeta \phi_1 \\ \phi_2
    \end{bmatrix}
\end{equation}

The boundary conditions lead to the following algebraic system:

\begin{equation} \label{eq:eigval_calc_4}
    \tilde{\Lambda} \begin{bmatrix}
        \phi_1 \\ \partial_\zeta \phi_1 \\ \phi_2
    \end{bmatrix}_{\zeta=0} = 0
\end{equation}

where $\tilde{\Lambda}$ is derived from the boundary conditions. The characteristic equation $det(\tilde{\Lambda}) = 0$ is solved numerically using the parameters in Table~\ref{tab:pars}. The resulting eigenvalue distribution is depicted in Figure [XXX].

The eigenvalues of $\mathfrak{A}$ and its adjoint $\mathfrak{A}^*$ are equivalent, confirming that $\mathfrak{A}$ exhibits Riesz-spectral properties, as shown in Figure [XXX].

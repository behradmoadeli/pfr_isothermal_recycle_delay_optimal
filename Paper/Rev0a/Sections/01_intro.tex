Many chemical, petrochemical, and biochemical unit operation processes are modelled as distributed parameter systems (DPS). When these processes are described using first-principle modeling, they result in a class of partial differential equations (PDEs) to effectively capture diffusion, transport, and reaction phenomena, leading to infinite-dimensional state space representations \autocite{ray1981advanced}. This characteristic presents significant challenges, making the control and estimation of DPS inherently more complex than finite-dimensional systems. Two primary methods have emerged for addressing DPS control. One is early lumping, which approximates the infinite-dimensional system with a finite-dimensional model \autocite{davison1976robust, francis1977linear}. While this method enables the use of standard regulator design techniques, mismatches between the dynamical properties of the original DPS and the approximate lumped parameter model can occur, negatively affecting the performance of the designed regulator \autocite{moghadam2012infinite}. The second method is late lumping, which directly tackles the infinite-dimensional system before applying numerical solutions. This approach introduces a challenging yet fertile direction of research, leading to many meaningful contributions that address various aspects of control and estimation of infinite-dimensional systems.

Among notable studies utilizing late lumping method for control of convection-reaction chemical systems resulting in first order hyperbolic PDEs, \Citeauthor{christofides1998robust} explored the robust control of quasi-linear first-order hyperbolic PDEs, providing explicit controller synthesis formulas for uncertainty decoupling and attenuation \autocite{christofides1998robust}. \Citeauthor{krstic2008backstepping} extended boundary feedback stabilization techniques for first-order hyperbolic PDEs using a backstepping method, converting the unstable PDE into a system for finite-time convergence \autocite{krstic2008backstepping}. Relevant applications of reaction-convection systems other than tubular reactors have also been addressed within this field, resulting in regulator/observer design strategies for chemical systems governed by first order hyperbolic PDEs. \Citeauthor{xu2016state} addressed the state feedback regulator problem for a countercurrent heat exchanger system, utilizing an infinite-dimensional approach to ensure that the controlled output tracks a reference signal \autocite{xu2016state}. Xie and Dubljevic \Citeauthor{xie2021discrete} developed a discrete-time output regulator for gas pipeline networks, emphasizing the transformation of continuous-time models into discrete-time systems while preserving essential continuous-time properties \autocite{xie2021discrete}. This work was further extended by \Citeauthor{zhang2023tracking}, who proposed a tracking model predictive control and moving horizon estimation design for pipeline systems, addressing the challenges of state and parameter estimation in an infinite-dimensional chemical system governed by first order hyperbolic PDEs \autocite{zhang2023tracking}. For a similar convection-reaction system, \Citeauthor{zhang2022dynamic} proposed a model predictive control strategy, incorporating a Luenberger observer to achieve output constrained regulation in a system modeled by nonlinear coupled hyperbolic PDEs \autocite{zhang2022dynamic}.

Additionally, diffusion-convection-reaction systems resulting in parabolic PDEs are also addressed in several works. For example, \Citeauthor{Christofides2012book} addressed order reduction methods for diffusion-convection-reaction type of reactors \autocite{Christofides2012book}. \Citeauthor{dubljevic2006predictive2} utilized modal decomposition to capture dominant modes of a DPS to construct a reduced order finite dimensional system, which enables the design of a low dimensional controller for a diffusion-convection-reaction type reactor described by second order parabolic PDEs \autocite{dubljevic2006predictive2}. \Citeauthor{ozorio2019heat} designed and compared the performance of a full-state and output feedback controller for a diffusion-convection heat exchanger system \autocite{ozorio2019heat}. In \Citeauthor{khatibi2021model}'s work, an axial dispersion tubular reactor equipped with recycle stream is considered as a second order parabolic DPS, with a predictive controller being utilized to optimally control the reactor \autocite{khatibi2021model}.  Although the presence of recycle is common in industrial reactor designs, this work is one of the few contributions in this field that addresses a diffusion-convection-reaction system equipped with a recycle stream.

Moreover, continuous-time optimal control design is a well-developed concept for distributed parameter systems, particularly when the system generator is either a self-adjoint operator or can be transformed into one through a proper linear transformation \autocite{morrisbook}. However, there are distributed parameter systems that do not possess this property. Instead, the system generator belongs to the domain of Riesz-spectral operators. Rather than an orthonormal basis for the function-space, these generators introduce a bi-orthonormal set of eigenfunctions as the basis. Optimal controller design for these systems was initially addressed in \Citeauthor{curtainbook} \autocite{curtainbook}. Since then, significant work has been done in this field. For instance, continuous-time optimal control design for a cracking catalytic reactor, another convection-reaction system governed by first-order hyperbolic PDEs, has been achieved by solving an operator Riccati equation (ORE)\autocite{aksikas2009lq}. This work has been further extended to time-varying PDEs of the same class\autocite{aksikas2013optimal}. The same approach has been applied to develop a full-state feedback\autocite{mohammadi2012lq} and output feedback\autocite{aksikas2024spectral} linear quadratic (LQ) optimal regulator for a boundary-controlled convection-reaction system, utilizing the properties of a Riesz-spectral generator for the system.

% [delay is another DPS. IO delay darim. State delay in other fields, vali chemical no application]
Book chapter reference \autocite{krstic2009book}

As stated previously, not much work is published addressing chemical reactors equipped with recycle as distributed parameter systems. Even in \Citeauthor{khatibi2021model}'s work, the recycle is assumed to be instantanous; a simplifying assumption that does not resonate well with reality. In fact, % can be a rare example state delay

% [present work focus: boundary output LQR, boundary conds, delayed recycle => Riesz + some refs => method overview]
%     The present work focuses on the development of an ORE-based LQ control strategy for a class of linear hyperbolic distributed  parameter systems interacting with a linear lumped parameter system through a Dirichlet boundary condition.
%     In such systems, the  boundary control actuation involves finite dimensional dynamics, i.e., the manipulated input acts through the lumped system on the  boundaries of the distributed system.
%     The paper’s main contributions can be summarized as follows:
%         first, the system under study is  escribed as an infinite-dimensional state-space by using the boundary control transformation method.
%         Then, dynamical properties of  he system including stability, stabilizability, and detectability are analysed.
%         Subsequently, the infinite-time horizon LQ control  roblem for the system is formulated, and the related ORE is converted to a set of matrix Riccati equations.
%         Finally, a  omputational algorithm is proposed for solving the resulting matrix Riccati equations.
%         To demonstrate the theory, an illustrative example is given.

% vase riesz: this is an example of folan, where although no orthogonal, but adjoint gives biorthogonal [refs]
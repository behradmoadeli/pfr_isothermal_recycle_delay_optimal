Many chemical, petrochemical, and biochemical unit operation processes are modeled as distributed parameter systems (DPS). When these processes are described using first-principle modeling, they result in a class of partial differential equations (PDEs) to effectively capture diffusion, transport, and reaction phenomena, leading to infinite-dimensional state space representations \autocite{ray1981advanced,christofides1998robust}. This characteristic presents significant challenges, making the control and estimation of DPS inherently more complex than finite-dimensional systems. Two primary methods have emerged for addressing DPS control. One is early lumping, which approximates the infinite-dimensional system with a finite-dimensional model \autocite{davison1976robust, francis1977linear}. While this method enables the use of standard regulator design techniques, mismatches between the dynamical properties of the original DPS and the approximate lumped parameter model can occur, negatively affecting the performance of the designed regulator \autocite{moghadam2012infinite}. The second method is late lumping, which directly tackles the infinite-dimensional system before applying numerical solutions. This approach introduces a challenging yet fertile direction of research, leading to many meaningful contributions that address various aspects of control and estimation of infinite-dimensional systems \autocite{dubljevic2006predictive1, dubljevic2006predictive2, xu2016state, ozorio2019heat, cassol2024chemostat}; to cite a few.

Many chemical, petrochemical, and biochemical unit operation processes are modeled as distributed parameter systems (DPS). When these processes are described using first-principle modeling, they result in a class of partial differential equations (PDEs) to effectively capture diffusion, transport, and reaction phenomena, leading to infinite-dimensional state space representations \autocite{ray1981advanced,christofides1998robust}. This characteristic presents significant challenges, making the control and estimation of DPS inherently more complex than finite-dimensional systems. Two primary methods have emerged for addressing DPS control. One is early lumping, which approximates the infinite-dimensional system with a finite-dimensional model \autocite{davison1976robust, francis1977linear}. While this method enables the use of standard regulator design techniques, mismatches between the dynamical properties of the original DPS and the approximate lumped parameter model can occur, negatively affecting the performance of the designed regulator \autocite{moghadam2012infinite}. The second method is late lumping, which directly tackles the infinite-dimensional system before applying numerical solutions. This approach introduces a challenging yet fertile direction of research, leading to many meaningful contributions that address various aspects of control and estimation of infinite-dimensional systems \autocite{dubljevic2006predictive1, dubljevic2006predictive2, xu2016state, ozorio2019heat, cassol2024chemostat}; to cite a few.

[Start counting optimal control works for reactor DPS (use alizade2013 for good refs. add some newer + some our group), finish by hamid's recycle. Brief explanation for each work.]

[present work focus: boundary output LQR, boundary conds, delayed recycle => Riesz + some refs => method overview]
    The present work focuses on the development of an ORE-based LQ control strategy for a class of linear hyperbolic distributed  parameter systems interacting with a linear lumped parameter system through a Dirichlet boundary condition.
    In such systems, the  boundary control actuation involves finite dimensional dynamics, i.e., the manipulated input acts through the lumped system on the  boundaries of the distributed system.
    The paper’s main contributions can be summarized as follows:
        first, the system under study is  escribed as an infinite-dimensional state-space by using the boundary control transformation method.
        Then, dynamical properties of  he system including stability, stabilizability, and detectability are analysed.
        Subsequently, the infinite-time horizon LQ control  roblem for the system is formulated, and the related ORE is converted to a set of matrix Riccati equations.
        Finally, a  omputational algorithm is proposed for solving the resulting matrix Riccati equations.
        To demonstrate the theory, an illustrative example is given.

vase riesz: this is an example of folan, where although no orthogonal, but adjoint gives biorthogonal [refs]
\section{Full-state Feedback Regulator}

The bi-orthogonal basis generated by the Riesz-spectral operator $\mathfrak{A}$ provides the foundation for solving the Operator Riccati Equation (ORE), a crucial step in the design of a Linear Quadratic Regulator (LQR). The objective is to derive a feedback control law that minimizes the infinite-time cost function, as defined in Equation~\ref{eq:cost_fun}, with $\mathfrak{Q}$ and $\mathfrak{R}$ being positive semi-definite operators that penalize state deviations and control actions, respectively.

\begin{equation} \label{eq:cost_fun}
    J(x_0, u) \int_0^{\infty} \langle x(s), \mathfrak{Q} x(s)\rangle + \langle u(s), \mathfrak{R} u(s)\rangle ds
\end{equation}

\subsection{Operator Riccati Equation}

The LQR problem is solved by finding the unique positive semi-definite operator $\mathbf{\Pi}$, which satisfies the ORE presented in Equation~\ref{eq:ORE_1}. This operator is then used to compute the feedback gain that ensures optimal control of the system.

\begin{equation} \label{eq:ORE_1}
    \langle \mathfrak{A}^* \mathbf{\Pi} x, y\rangle + \langle \mathbf{\Pi} \mathfrak{A} x, y \rangle - \langle \mathbf{\Pi} \mathfrak{B} \mathfrak{R}^{-1} \mathfrak{B}^* \mathbf{\Pi} x, y\rangle + \langle \mathfrak{Q} x, y\rangle = 0
\end{equation}

Given that the solution to the ORE is unique for any set of functions in the domain of operator $\mathfrak{A}$, we can select $x = \phi_m$ and $y = \phi_n$, i.e. the eigenfunctions of $\mathfrak{A}$. Applying this choice, and noting that $\mathbf{\Pi}$ is self-adjoint, leads to the simplified Equation~\ref{eq:ORE_2}.

\begin{equation} \label{eq:ORE_2}
    \langle \mathbf{\Pi} \phi_m, \mathfrak{A} \phi_n \rangle
    + \langle \mathfrak{A} \phi_m, \mathbf{\Pi} \phi_n \rangle
    - \mathfrak{R}^{-1} \langle \mathfrak{B}^* \mathbf{\Pi} \phi_m, \mathfrak{B}^* \mathbf{\Pi} \phi_n \rangle 
    + \langle \mathfrak{Q} \phi_m, \phi_n \rangle = 0
\end{equation}

To ensure that the domain and range of $\mathbf{\Pi}$ match those of $\mathfrak{A}$ and $\mathfrak{A}^*$, respectively, $\mathbf{\Pi}$ can be expressed as an infinite sum, as shown in Equation~\ref{eq:P}. The coefficients $p_{i,j}$ can be interpreted as elements of an infinite-dimensional matrix $\tilde{P}$, which represents the operator $\mathbf{\Pi}$. This forms the first step in converting the ORE to the corresponding Matrix Riccati Equation (MRE).

\begin{equation} \label{eq:P}
    \mathbf{\Pi} x = \sum_{i=1}^{\infty}\sum_{j=1}^{\infty} p_{i,j} \langle x, \psi_j \rangle \psi_i \qquad
    \forall {i,j}: \quad p_{i,j} \in \mathbb{C}
\end{equation}

\subsection{Obtaining $\mathfrak{B}$ and $\mathfrak{B}^*$}

Before further simplifying the ORE, it is essential to define the operators $\mathfrak{B}$ and $\mathfrak{B}^*$. Given the boundary-control nature of the system, $\mathfrak{B}$ is defined to properly project the control input $u \in \mathbb{R}^1$ onto the state space $X: L^2[0,1] \times L^2[0,1]$, as outlined in Equation~\ref{eq:B}.

\begin{equation} \label{eq:B}
    \mathfrak{B} u \equiv
    \begin{bmatrix}
        \delta(\zeta) \\ 0
    \end{bmatrix} \cdot u
\end{equation}

The adjoint operator $\mathfrak{B}^*$ is obtained by leveraging the properties of $\mathfrak{A}$ and $\mathfrak{A}^*$, i.e. their expressions as well as their domains (as shown in Equations~\ref{eq:operator_A}~and~\ref{eq:adjoint_A}), after applying integration by parts to the result of the inner products, as summarized in Equation~\ref{eq:B*}.

\begin{equation} \label{eq:B*}
    \begin{aligned}
        \langle \mathfrak{A} x + \mathfrak{B} u, y \rangle
        &= \langle \mathfrak{A} x, y \rangle
        + \langle \mathfrak{B} u, y \rangle
        = \langle x, \mathfrak{A}^* y\rangle
        + \langle u, \mathfrak{B}^* y \rangle \\
        \langle u, \mathfrak{B}^* y \rangle
        &= \langle \mathfrak{A} x + \mathfrak{B} u, y \rangle
        - \langle x, \mathfrak{A}^* y\rangle
        \Rightarrow \hdots \\ \Rightarrow \mathfrak{B}^* (.) &= \Bigl[ v(1-R) \int_0^1 \delta(\zeta) (.) d\zeta \quad , \quad 0 \Bigr]
    \end{aligned}
\end{equation}

\subsection{Matrix Riccati Equation}

Using the expression for $\mathbf{\Pi}$ in Equation~\ref{eq:P}, along with the derived $\mathfrak{B}^*$ from Equation~\ref{eq:B*}, and the eigenvalue problem $\mathfrak{A}\phi_i = \lambda_i \phi_i$, the ORE can be reformulated as the Matrix Riccati Equation (MRE) shown in Equation~\ref{eq:MRE}. Here, $\gamma_i \equiv v(1-R) \left. \psi_{1}^{(i)} \right|_{\zeta = 0}$, and $q_{m,n} = \langle \mathfrak{Q} \phi_m, \phi_n \rangle$.

\begin{equation}\label{eq:MRE}
    p_{n,m} (\lambda_m + \overline{\lambda_n})
    - \mathfrak{R}^{-1} \langle \sum_i p_{i,m} \gamma_i, \sum_i p_{i,n} \gamma_i \rangle
    + q_{m,n} = 0
\end{equation}

Due to the infinite-dimensional nature of $\tilde{P}$, a numerical solution is impractical. This challenge is addressed by selecting the first $N$ eigenmodes of the system as its dominant modes. This means truncating the infinite sums in the MRE and reducing the infinite-dimensional system to a finite set of nonlinear algebraic equations that can be solved to obtain an equivalent $N \times N$ matrix $P$, i.e. a truncated approximation of matrix $\tilde{P}$. The optimal full-state feedback gain is then calculated using Equation~\ref{eq:fullstate_gain}, ensuring closed-loop stability.

\begin{equation} \label{eq:fullstate_gain}
    \begin{aligned}
        u(t) &= \langle k (\zeta), x(\zeta, t) \rangle = \mathfrak{B}^* \mathbf{\Pi} x(\zeta, t) \\
        &= \sum_{i=1}^N\sum_{j=1}^N p_{i,j} \langle x(\zeta, t), \psi_j(\zeta) \rangle \gamma_i \\
        &= \sum_{i=1}^N\sum_{j=1}^N p_{i,j} \gamma_i \int_0^1 x(\zeta, t) \cdot \overline{\psi_j}(\zeta) d\zeta \\
        &= \int_0^1 \sum_{i=1}^N\sum_{j=1}^N p_{i,j} \gamma_i \overline{\psi_j}(\zeta) \cdot x(\zeta, t) d\zeta \\
        \Rightarrow k(\zeta) &\equiv \sum_{i=1}^N\sum_{j=1}^N p_{i,j} \gamma_i \overline{\psi_j}(\zeta)
    \end{aligned}
\end{equation}

The computed gain is a function of space and is calculated offline. The control action at any given time instance is the inner product of this gain with the current state of the system, thus justifying the term ``full-state'' feedback.

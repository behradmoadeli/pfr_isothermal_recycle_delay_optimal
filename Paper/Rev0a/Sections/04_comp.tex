\subsection{Output Feedback Compensator}

In the previous work, the optimal regulator was designed under the assumption that it had full access to the system's states, utilizing the inner product of the feedback gain and the states. However, this assumption is not feasible in realistic applications. To address this, an observer is introduced to estimate and reconstruct the states by measuring the system's output in real time. The output, in this context, is taken as the concentration at the reactor outlet, as defined in Equation~\ref{eq:BC}. This leads to the definition of the output operator $\mathfrak{C}$ in the linear time-invariant (LTI) system $\mathbf{\Sigma(\mathfrak{A},\mathfrak{B},\mathfrak{C},-)}$, which is subsequently used to determine the observer gains. The formulation is shown in Equation~\ref{eq:C}:

\begin{equation} \label{eq:C}
    \mathfrak{C} \equiv \begin{bmatrix}
        \int_0^1 \delta(\zeta-1) (.) d\zeta \quad , \quad 0
    \end{bmatrix}
\end{equation}

where $\delta(\zeta)$ denotes the Dirac delta function. Regarding the choice of observer, Luenberger-based observers are well-suited for infinite-dimensional systems when the system parameters are perfectly known \autocite{ali2015reviewobserver}. Among the various methods to compute the gain for this class of observers, pole-placement is a solid, straightforward, and reliable approach for state reconstruction. To ensure that the state reconstruction dynamics converge more quickly than the regulation dynamics, the poles of the observer-based controller are placed to the left of the poles of the full-state feedback controller. This practice is common in the design of observer-based controllers for infinite-dimensional systems \autocite{morrisbook}.

(Sample Diagram)

\begin{tikzpicture}[auto, node distance=2cm,>=latex]
    % Define block styles
    \tikzstyle{block} = [draw, fill=white, rectangle, minimum height=3em, minimum width=3em]
    \tikzstyle{sum} = [draw, fill=white, circle, node distance=1.5cm]
    \tikzstyle{input} = [coordinate]
    \tikzstyle{output} = [coordinate]
    
    % Nodes
    \node [input, name=input] {};
    \node [sum, right of=input] (sum) {};
    \node [block, right of=sum] (B) {$B$};
    \node [block, right of=B] (int) {$\int$};
    \node [block, right of=int] (C) {$C$};
    \node [output, right of=C] (output) {};
    \node [block, below of=B] (A) {$A$};
    \node [block, below of=A] (K) {$-K$};
  
    % Connections
    \draw [draw,->] (input) -- node {$u$} (sum);
    \draw [->] (sum) -- (B);
    \draw [->] (B) -- (int);
    \draw [->] (int) -- (C);
    \draw [->] (C) -- node [name=y] {$y$} (output);
    \draw [->] (int) |- (A);
    \draw [->] (A) -| (sum);
    \draw [->] (A) -- (K);
    \draw [->] (K) -- (sum);
  \end{tikzpicture}

% To implement this, the eigenvalues of the closed-loop system with large real parts were sufficiently shifted to the left of the complex plane, ensuring that the maximum real parts of the observer-based poles satisfy Equation~\ref{eq:obsv_pole}.

% \begin{equation} \label{eq:obsv_pole}
%     \forall i: \quad
%     \mathfrak{Re}(\lambda_i^{\mathrm{obsv}}) \leq 3 \times
%     \max{\Biggl\{
%         \mathfrak{Re}
%             \Bigl(
%                 \sigma\bigl(
%                     \mathfrak{A}-\mathfrak{B}k
%                 \bigr)
%             \Bigr)
%         \Biggr\}}
% \end{equation}

% where $\lambda_i^{\mathrm{obsv}}$ is the $i^{\text{th}}$ eigenvalue of the observer-based regulator, and $\sigma\bigl(\mathfrak{A}-\mathfrak{B}k\bigr)$ refers to the spectrum of the closed-loop system quipped with full-state feedback. This will guarantee the desired faster state reconstruction dynamics compared to regulator stabilization dynamics.

% \begin{itemize}
%     \item Briefly discuss the performance of full-state feedback in control systems.
%     \item Highlight the limitations and challenges associated with full-state feedback.
%     \item Introduce the concepts of state reconstruction and compensator design.
%     \item Present a block diagram illustrating the full-state regulation and compensator approach.
%     \item Define the system output and the output matrix \( C \).
%     \item Explore the criteria for choosing an appropriate observer.
%     \item Describe the method used for calculating the observer gain \( L \) and justify the choice.
%     \item Explain how the observer gain \( L \) is implemented in the control system.
% \end{itemize}

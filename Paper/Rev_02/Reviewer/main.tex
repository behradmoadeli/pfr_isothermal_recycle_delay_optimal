\documentclass[10pt,answers]{exam}


\usepackage{graphicx}
\usepackage{amsmath}
\usepackage{color}
\usepackage{amssymb}
\usepackage{setspace}
\usepackage{epstopdf}
\usepackage{empheq}
\setlength{\marginparwidth}{2cm}
\usepackage{easyReview} % \alert, \highlight, \remove, \add, \replace, \comment
\usepackage{subcaption}
% \usepackage[letterpaper]{geometry}
% \usepackage[definethebibliography]{easybib}
\usepackage{float}
\renewcommand{\solutiontitle}{\noindent\textbf{Authors Reply:}\par\noindent}

\begin{document}
\section*{Reply to Associate Editor and Reviewers}
We are grateful for your and other reviewers' critical comments and suggestions. 
\vspace{0.2in}
\\
The authors appreciate the reviewers' comments, and the appropriate corrections have been made to the manuscript. All changes in the text are emphasized in a different text color (red) in the highlighted manuscript file that was uploaded as a {\bf{``Supplementary Material for Review but Not for Publication''}}. The major changes to the manuscript are detailed below:\\
\begin{itemize}
\item{The abstract and introduction were rewritten to highlight the novelty of the contribution}
\item{An extended explanation of the parameter screening and optimization method was added.}
%\item{Figure 3;}
\end{itemize}

Also, please, find our point-by-point responses in the next pages.\\
\\ 
Sincerely yours,\\
Farzin Sadehlari, Guilherme Ozorio Cassol and Stevan Dubljevic
\newpage



\section*{Reviewer 2}

We appreciate the reviewer's comments and suggestions. The comments of Reviewer 2, along with our responses to each comment, are included below:

\begin{quote}
    ``The problem presented in the article is well-posed and written. The purpose is to present a novel approach to address the problem of intrinsic delay when there is a recycle stream in a process. However, I recommend submitting it to a journal focused on Control. This opinion is based on the following concerns regarding the process of a chemical reactor with recycle:''
\end{quote}


\begin{questions}

    \question Why is relevant to consider Danckwerts boundary conditions for the problem of control? Have you compared your results with those obtained considering other boundary conditions?

    \begin{solutionorbox}
        The title has been revised based on the suggested terminology.
    \end{solutionorbox}


    \question Concerning the recycle stream, have you studied the effect of the R, the recycle ratio, on your results? Please comment.

    \begin{solutionorbox}
        The title has been revised based on the suggested terminology.
    \end{solutionorbox}

    
    \question The authors used as the case study the problem of an axial dispersion tubular reactor incorporating diffusion, convection, and a first-order irreversible chemical reaction described by equations (1)-(2). While this is sufficient to present their approach, it is far from being extended to the more general problem, non-isothermal, and with more general kinetics such as biochemical or catalytic.

    \begin{solutionorbox}
        The title has been revised based on the suggested terminology.
    \end{solutionorbox}


    \question I recommend submitting it to a journal focused on Control.

    \begin{solutionorbox}
        The title has been revised based on the suggested terminology.
    \end{solutionorbox}

\end{questions}

% Reviewing: 3
\newpage
\section*{Reviewer 3}

We appreciate the reviewer's comments and suggestions. The comments of Reviewer 3, along with our responses to each comment, are included below:

\begin{quote}
    ``In this work, the authors address the optimal control of an axial tubular reactor with a recycle stream. They model the intrinsic time delay from the recycling process using a system of coupled parabolic and hyperbolic partial differential equations. The control input is applied at the inlet, and a continuous-time optimal linear quadratic regulator is designed to stabilize the system. Numerical simulations indicate effective full-state feedback regulator and observer-based regulator. This work presents an interesting methodology but there are some minor considerations that the authors need to address before this article can be published:''
\end{quote}

\begin{questions}

    \question \textbf{Major comment: } In section 4, the authors indicate that they discretized each state in space using 100 grid points. They must indicate if those points are equidistributed and how they came up with such discretization grid. Note that multiple works \cite{palma2023selection, assassa2016optimality, chen2014bilevel} have demonstrated that the selection and distribution of the discretization grid plays a crucial role in the computation of optimal control laws, i.e., a control law can be claimed to be optimal or not using the criterion of the Pontryagin's Minimum Principle (PMP). The reviewer recommends to assess the criterion of the Hamiltonian function (i.e., the PMP) to demonstrate that the discretization implemented is accurate and the solutions obtained for the control are optimal.

    \begin{solutionorbox}
        The title has been revised based on the suggested terminology.
    \end{solutionorbox}


    \question The manuscript lacks conclusions or further discussion about the control trajectories' results, such as the quality of the control actions or improvements in process operation (e.g., avoidance of constraint violations, disturbance rejection, etc.). Although the authors included several figures illustrating the process dynamics and control trajectories, these are not discussed in sufficient depth, i.e., avoid leaving the reader to draw their own conclusions from the figures. Additionally, the reviewer recommends reducing the number of figures, which could allow more space for further discussion of the results.

    \begin{solutionorbox}
        The title has been revised based on the suggested terminology.
    \end{solutionorbox}


    \question In section 3.1.3, the authors present the values for parameters R and D but provide no further details about the model's sensitivity to these parameters. Please include a detailed explanation of how these values were selected and discuss any potential limitations if the parameters are chosen incorrectly.

    \begin{solutionorbox}
        The title has been revised based on the suggested terminology.
    \end{solutionorbox}


    \question In section 4, the authors indicate that the process model was discretized in time and space, however, they mention that they obtained a system of ordinary differential equations (ODEs). Please clarify how this full discretization resulted in a system of ODEs.

    \begin{solutionorbox}
        The title has been revised based on the suggested terminology.
    \end{solutionorbox}


    \question The manuscript has some typos that the authors must correct, e.g., …setting for of distributed…, Then two full-state… In page 5, the acronym PIDEs is not previously defined. For section 4.3, the reviewer recommends to modify the expression “Last but not least” aiming not loose the formality of the manuscript.

    \begin{solutionorbox}
        The title has been revised based on the suggested terminology.
    \end{solutionorbox}


    \question The reviewer recommends to include a tables of nomenclature

    \begin{solutionorbox}
        The title has been revised based on the suggested terminology.
    \end{solutionorbox}
\end{questions}

% Comments to the Author
% In this work, the authors address the optimal control of an axial tubular reactor with a recycle stream. They model the intrinsic time delay from the recycling process using a system of coupled parabolic and hyperbolic partial differential equations. The control input is applied at the inlet, and a continuous-time optimal linear quadratic regulator is designed to stabilize the system. Numerical simulations indicate effective full-state feedback regulator and observer-based regulator. This work presents an interesting methodology but there are some minor considerations that the authors need to address before this article can be published:

% Major comment:
% 1) In section 4, the authors indicate that they discretized each state in space using 100 grid points. They must indicate if those points are equidistributed and how they came up with such discretization grid. Note that multiple works [] have demonstrated that the selection and distribution of the discretization grid plays a crucial role in the computation of optimal control laws, i.e., a control law can be claimed to be optimal or not using the criterion of the Pontryagin’s Minimum Principle (PMP). The reviewer recommends to assess the criterion of the Hamiltonian function (i.e., the PMP) to demonstrate that the discretization implemented is accurate and the solutions obtained for the control are optimal.



% Suggested Bibliography
% [1] Chen, W., Shao, Z., & Biegler, L. T. (2014). A bilevel NLP sensitivity‐based decomposition for dynamic optimization with moving finite elements. AIChE Journal, 60(3), 966-979.
% [2] Assassa, F., & Marquardt, W. (2016). Optimality-based grid adaptation for input-affine optimal control problems. Computers & chemical engineering, 92, 189-203.
% [3] Palma‐Flores, O., & Ricardez‐Sandoval, L. A. (2023). Selection and refinement of finite elements for optimal design and control: A Hamiltonian function approach. AIChE Journal, 69(5), e18009.


\newpage
\bibliographystyle{vancouver}
\bibliography{references.bib}
\end{document}